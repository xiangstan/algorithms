\documentclass[11pt]{article}  
\usepackage[margin=1in]{geometry}
\parindent=0in
\parskip=8pt
\usepackage{fancyhdr, amssymb, amsmath, graphicx, listings, float, subfig, enumerate, epstopdf, color, multirow, setspace, bm, textcomp, soul}
\usepackage[usenames,dvipsnames]{xcolor}
\usepackage{hyperref}

\pagestyle{fancy}
\lstset{
  basicstyle=\ttfamily,
  columns=fullflexible,
  frame=single,
  breaklines=true,
  postbreak=\mbox{\textcolor{red}{$\hookrightarrow$}\space},
}


\begin{document} 

\lhead{Project \# 1}
\chead{Xiang S Tan}
\rhead{\today}

\begin{center}\begin{Large}
CS 4720/5720 Design and Analysis of Algorithms

Project \#1

Student: Xiang S Tan
\end{Large}
\end{center}


\section*{Answers to homework problems:}

\begin{enumerate}
	\item Problem Statement: Knapsack Problem
		\begin{enumerate}[(a)]
		\item A total of n total items $i \in \{1, 2, ... n\}$
		\item Each item has its own value $i \in \{v_1, v_2, ... v_n\}$
		\item Each item has its own weight $i \in \{w_1, w_2, ... w_n\}$
		\item Create a bucket to pick a combination of items, which the total weight cannot exceed $W>0$
		\end{enumerate}
		
	\item  \textbf{Deliverable 1}: Create an exhaustive search algorithm in psueodocode and determine its time complexity
	 in term of $\Theta (.)$
	\begin{enumerate}[(a)]
		\item
		\textbf{My Approach} \\\\
		My first approach was to create an algorithm that loop through the entire item list
		 $i \in \{1, 2, ... n\}$. Start with every single items and try added the next items into
		 the bucket until the total weight in the bucket over the W. \\\\
		However, one requirement, "\textbf{all possible subsets of items}", makes my approach
		 difficult to implement.  
			\begin{enumerate}[(i)]
				\item
				Suppose I put item 1 in the bucket.
				\item
				Then item 2. The total weight of item 1 and item 2 is below the weight limit.
				\item
				Adding item 3, and the total weight does not exceed the limit.
				\item
				Adding item 4, and the total weight exceeds the limit.
				\item
				Then I skip item 4 and move to item 5. Item 5 fit in without break the limit.
				\item
				Now the bucket has item 1, item 2, item 3, and item 5.
				\item
				At this point, I do not know if item 1, item 3, item 4 and item 5, or item 1, item
				 3, item 4 and item 5 are over the weight limit.
			\end{enumerate}
		I could add another loop starting the second element against the first element in the
		 parent loop. However, my approach cannot guarantee all possible combination. \\\\
		Chatting with classmates, I have found a Python library called \textbf{itertools}. It
		contains a function called \textbf{\textit{combinations}}. It generate all unique
		combination of all elements within a given size of elements. \\\\
		I can use this function to generate all possible subsets of items and check each subset 
		 against the weight limit. \\\\

		\item
		\textbf{Pseudocode}

		\lstset{caption={Pseudocode for Brute Force}}
		\lstset{label={lst:code_direct}}
		\lstset{basicstyle=\footnotesize}
		\begin{lstlisting}[xleftmargin=\dimexpr-\leftmarginii-\leftmargini]
bestvalue = 0
bestweight = 0
lists = Generate Unique Combination
for i in lists :
	sum = sum(i)
	value = value of sum(i)
	if sum < weightlimit :
		bestvalue = value
		bestweight = sum
return {bestvalue, bestweight}
		\end{lstlisting}

		\item
		\textbf{Analysis} \\\\
		The order of growth of permutation is $n!$. a brute force of checking all subsets within 
		 the return value of \textbf{\textit{itertools.combinations}} contains two loops, which are 
		 \textbf{for i in lists} and \textbf{sum(i)}. The order of growth is $n^2$. \\\\
		$n!+n^2$ gives us $n!$. \\\\
		If we treat \textbf{\textit{itertools.combinations}} as a third-party function, and ignore
		 its order of growth, the total order of growth is $n^2$. \\\\
		Therefore, \hl{$T(n) \in \Theta (n!) \; or \; T(n) \in \Theta (n^2)$ without third-party  functions}. \\\\
		I wanted to tried reduce the order of growth. However, due to the time constraint (due
		 2021/09/26), I have to settle down with this approach. \\
	\end{enumerate}
	
	\item
	\textbf{Deliverable 2}: Create a greedy algorithm that always pick the highest value item first; and write 
	 the pseudocode and determine the efficiency class in terms of $\Theta (.)$
	\begin{enumerate}[(a)]
		\item
		\textbf{My Approach} \\\\
		This is is fairly straight forward. I need to reorder the given item list from the highest
		 value to least. Then reflect the index update with the weight list. \\\\
		During the processing of checking the total weight. I know the left most element has the
		 highest value; then add it in the bucket and check the total weight against the weight
		 limit. \\\\
		If adding the current item exceeds the weight limit, skip it and move to the next item
		 until all items are added into the bucket or no more items can fit in the bucket without
		 go over the limit. \\\\

		\item
		\textbf{Pseudocode}

		\lstset{caption={Pseudocode for Brute Force}}
		\lstset{label={lst:code_direct}}
		\lstset{basicstyle=\footnotesize}
		\begin{lstlisting}[xleftmargin=\dimexpr-\leftmarginii-\leftmargini]
value = 0
weight = 0
lists = Sort the Given Item List
for i in lists :
	temp = weight + weightlist[i]
	if temp < weightlimit :
		value = value + valuelist[i]
		weight = weight + weightlist[i]
return {value, weight}
		\end{lstlisting}

		\item
		\textbf{Analysis} \\\\
		There are two parts of this approach. Sorting array has an order of growth of $n^2$;
		 and adding weight of each item together is another $n$. \\\\
		The order of growth of $n^2 + n$ is $n^2$, or treat sorting array function
		 \textbf{\textit{np.argsort}} as a third-party function, $n$. \\\\
		Therefore, \hl{$T(n) \in \Theta (n^2) \; or \; T(n) \in \Theta (n)$ without third-party functions}. \\\\
	\end{enumerate}

	\item
	\textbf{Deliverable 3}: For the greedy algorithm: provide a proof of correctness or a counterexample with demonstrates
	 its possible suboptimality. \\\\
	We have talked about this in class. \\
	Suppose that the total capacity is 10, and four items are given $\{v_1 = 42, w_1 = 7\}, \{v_2 = 12, w_2 = 3\}, \{v_3 = 40, w_3 = 4\}, and \; \{v_4 = 24, w_4 = 5\}$. \\\\
	By the definition of the greedy algorithm, I first resort the item list by with value. \\\\
	We have $\{v_1 = 42, w_1 = 7\}, \{v_3 = 40, w_3 = 4\}, \{v_4 = 24, w_4 = 5\}, and \; \{v_2 = 12, w_2 = 3\}$. \\\\
	Then I put items in the bucket. From the most valuable items to least, I get item 1 and item 
	 2 without exceeding the total capacity. The total value is:\\\\
	 Total Value: $v_1+v_2=42+12=54$ and Total Weight: $w_1+w_2=7+3=10$. \\\\
	Clearly I can see that putting item 3 and item 4 in the bucket will give me the total value: \\\\
	Total Value: $v_3+v_4=40+24=64$ and Total Weight: $w_3+w_4=4+5=9$. \\\\
	I can see that putting item 3 and item 4 into the buck is a better choice than item 1 and item 2. \\\\
	Therefore, \hl{Greedy Algorithm is not a most optimal choice}.\\\\

	\item
	\textbf{Deliverable 4}: Implement both exhaustive and greedy algorithms in programming language of choice.
	\begin{enumerate}[(a)]
	\item W = 10000
	\item n from 3 to 50
	\item Generate 5 random inputs to each n.
	\item Generate three scatter plots:
		\begin{enumerate}[(i)]
		\item The time (or number of steps) it takes for each of the two algorithms to terminate.
		\item The total weight of items by each algorithm.
		\item The total chosen by the exhaustive algorithm, \textit{divided} by the total value chosen by the greedy algorithm.
		\end{enumerate}
	\end{enumerate}
	
	\begin{figure}[h]
	\includegraphics[width=\textwidth]{./figure_1.png}
	\end{figure}
	
	I have tried to run $n>22$. However, it seemed that the program runs too long to complete. I have to drop it back to 22. \\\\
	Top left figure illustrates the executed time of each algorithm. Blue color is the data from exhaustive search and green color is the greedy search. \\\\
	My analysis showed that the exhaustive algorithm had an order of growth of $T(n) \in \Theta (n^2)$ without third-party functions. However, the plot did not show any growth in term of n vs time to execute the program.\\\\
	However, I mentioned that the program took too long to run when $n>22$. The order of growth should not be as same as $\Theta (1)$. \\\\
	Top right figure shows n vs the total weight. The total capacity is set to be 10,000. Therefore, no results are exceeding the limitation. There are two data points shows below 8000 and 5 points below 8500 of greedy algorithm, whereas exhaustive algorithm only has 1 point in each category.\\\\
	From \textbf{Deliverable 3}, I understand that the greedy algorithm is not the most optimal choice.\\\
	The bottom plot showed a similar result. Exhaustive algorithm gives a better result in term of picking the "best" value of each random item list.
	 
\end{enumerate}

\end{document}
